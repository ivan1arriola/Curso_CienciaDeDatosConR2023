% Options for packages loaded elsewhere
\PassOptionsToPackage{unicode}{hyperref}
\PassOptionsToPackage{hyphens}{url}
%
\documentclass[
]{article}
\usepackage{amsmath,amssymb}
\usepackage{lmodern}
\usepackage{iftex}
\ifPDFTeX
  \usepackage[T1]{fontenc}
  \usepackage[utf8]{inputenc}
  \usepackage{textcomp} % provide euro and other symbols
\else % if luatex or xetex
  \usepackage{unicode-math}
  \defaultfontfeatures{Scale=MatchLowercase}
  \defaultfontfeatures[\rmfamily]{Ligatures=TeX,Scale=1}
\fi
% Use upquote if available, for straight quotes in verbatim environments
\IfFileExists{upquote.sty}{\usepackage{upquote}}{}
\IfFileExists{microtype.sty}{% use microtype if available
  \usepackage[]{microtype}
  \UseMicrotypeSet[protrusion]{basicmath} % disable protrusion for tt fonts
}{}
\makeatletter
\@ifundefined{KOMAClassName}{% if non-KOMA class
  \IfFileExists{parskip.sty}{%
    \usepackage{parskip}
  }{% else
    \setlength{\parindent}{0pt}
    \setlength{\parskip}{6pt plus 2pt minus 1pt}}
}{% if KOMA class
  \KOMAoptions{parskip=half}}
\makeatother
\usepackage{xcolor}
\usepackage[margin=1in]{geometry}
\usepackage{color}
\usepackage{fancyvrb}
\newcommand{\VerbBar}{|}
\newcommand{\VERB}{\Verb[commandchars=\\\{\}]}
\DefineVerbatimEnvironment{Highlighting}{Verbatim}{commandchars=\\\{\}}
% Add ',fontsize=\small' for more characters per line
\usepackage{framed}
\definecolor{shadecolor}{RGB}{248,248,248}
\newenvironment{Shaded}{\begin{snugshade}}{\end{snugshade}}
\newcommand{\AlertTok}[1]{\textcolor[rgb]{0.94,0.16,0.16}{#1}}
\newcommand{\AnnotationTok}[1]{\textcolor[rgb]{0.56,0.35,0.01}{\textbf{\textit{#1}}}}
\newcommand{\AttributeTok}[1]{\textcolor[rgb]{0.77,0.63,0.00}{#1}}
\newcommand{\BaseNTok}[1]{\textcolor[rgb]{0.00,0.00,0.81}{#1}}
\newcommand{\BuiltInTok}[1]{#1}
\newcommand{\CharTok}[1]{\textcolor[rgb]{0.31,0.60,0.02}{#1}}
\newcommand{\CommentTok}[1]{\textcolor[rgb]{0.56,0.35,0.01}{\textit{#1}}}
\newcommand{\CommentVarTok}[1]{\textcolor[rgb]{0.56,0.35,0.01}{\textbf{\textit{#1}}}}
\newcommand{\ConstantTok}[1]{\textcolor[rgb]{0.00,0.00,0.00}{#1}}
\newcommand{\ControlFlowTok}[1]{\textcolor[rgb]{0.13,0.29,0.53}{\textbf{#1}}}
\newcommand{\DataTypeTok}[1]{\textcolor[rgb]{0.13,0.29,0.53}{#1}}
\newcommand{\DecValTok}[1]{\textcolor[rgb]{0.00,0.00,0.81}{#1}}
\newcommand{\DocumentationTok}[1]{\textcolor[rgb]{0.56,0.35,0.01}{\textbf{\textit{#1}}}}
\newcommand{\ErrorTok}[1]{\textcolor[rgb]{0.64,0.00,0.00}{\textbf{#1}}}
\newcommand{\ExtensionTok}[1]{#1}
\newcommand{\FloatTok}[1]{\textcolor[rgb]{0.00,0.00,0.81}{#1}}
\newcommand{\FunctionTok}[1]{\textcolor[rgb]{0.00,0.00,0.00}{#1}}
\newcommand{\ImportTok}[1]{#1}
\newcommand{\InformationTok}[1]{\textcolor[rgb]{0.56,0.35,0.01}{\textbf{\textit{#1}}}}
\newcommand{\KeywordTok}[1]{\textcolor[rgb]{0.13,0.29,0.53}{\textbf{#1}}}
\newcommand{\NormalTok}[1]{#1}
\newcommand{\OperatorTok}[1]{\textcolor[rgb]{0.81,0.36,0.00}{\textbf{#1}}}
\newcommand{\OtherTok}[1]{\textcolor[rgb]{0.56,0.35,0.01}{#1}}
\newcommand{\PreprocessorTok}[1]{\textcolor[rgb]{0.56,0.35,0.01}{\textit{#1}}}
\newcommand{\RegionMarkerTok}[1]{#1}
\newcommand{\SpecialCharTok}[1]{\textcolor[rgb]{0.00,0.00,0.00}{#1}}
\newcommand{\SpecialStringTok}[1]{\textcolor[rgb]{0.31,0.60,0.02}{#1}}
\newcommand{\StringTok}[1]{\textcolor[rgb]{0.31,0.60,0.02}{#1}}
\newcommand{\VariableTok}[1]{\textcolor[rgb]{0.00,0.00,0.00}{#1}}
\newcommand{\VerbatimStringTok}[1]{\textcolor[rgb]{0.31,0.60,0.02}{#1}}
\newcommand{\WarningTok}[1]{\textcolor[rgb]{0.56,0.35,0.01}{\textbf{\textit{#1}}}}
\usepackage{graphicx}
\makeatletter
\def\maxwidth{\ifdim\Gin@nat@width>\linewidth\linewidth\else\Gin@nat@width\fi}
\def\maxheight{\ifdim\Gin@nat@height>\textheight\textheight\else\Gin@nat@height\fi}
\makeatother
% Scale images if necessary, so that they will not overflow the page
% margins by default, and it is still possible to overwrite the defaults
% using explicit options in \includegraphics[width, height, ...]{}
\setkeys{Gin}{width=\maxwidth,height=\maxheight,keepaspectratio}
% Set default figure placement to htbp
\makeatletter
\def\fps@figure{htbp}
\makeatother
\setlength{\emergencystretch}{3em} % prevent overfull lines
\providecommand{\tightlist}{%
  \setlength{\itemsep}{0pt}\setlength{\parskip}{0pt}}
\setcounter{secnumdepth}{-\maxdimen} % remove section numbering
\usepackage{bbm}
\usepackage{amsthm}
\usepackage{amsmath}
\usepackage[spanish]{babel}
\ifLuaTeX
  \usepackage{selnolig}  % disable illegal ligatures
\fi
\IfFileExists{bookmark.sty}{\usepackage{bookmark}}{\usepackage{hyperref}}
\IfFileExists{xurl.sty}{\usepackage{xurl}}{} % add URL line breaks if available
\urlstyle{same} % disable monospaced font for URLs
\hypersetup{
  pdftitle={Tarea 1},
  pdfauthor={Ivan Arriola 55366796},
  hidelinks,
  pdfcreator={LaTeX via pandoc}}

\title{Tarea 1}
\usepackage{etoolbox}
\makeatletter
\providecommand{\subtitle}[1]{% add subtitle to \maketitle
  \apptocmd{\@title}{\par {\large #1 \par}}{}{}
}
\makeatother
\subtitle{Ciencia de Datos con R}
\author{Ivan Arriola 55366796}
\date{Entrega 12/4 23:59 PM}

\begin{document}
\maketitle

La fecha para entregar la Tarea 1 es el 12 de Abril a las 23:59 PM. La
tarea es individual por lo que cada uno tiene que escribir su propia
versión de la misma aunque se incentiva la consulta de dudas con
estudiantes del curso así como en foros de EVA.

La tarea debe ser realizada en RMarkdown disponible en tu repositorio de
GitHub (generado en la Actividad 5) donde van a ir poniendo todas las
tareas y actividades del curso en diferentes carpetas y cada uno con su
correspondiente proyecto de RStudio.

En el \textbf{YAML} del .Rmd incluí tu nombre y CI (cambiar donde dice
author: ``Alumno'').

\textbf{MUY IMPORTANTE} Cuando generen el proyecto de RStudio siempre
revisar que están utilizando la codificación de texto UTF-8 (text
encoding UTF8). Para ello, debes ir a:

\textbf{Tools -\textgreater{} Project Options -\textgreater{} Code
Editing -\textgreater{} Text encoding seleccionar UTF-8}.

El repositorio de GitHub para esta tarea debe contener el únicamente
archivo .Rmd con la solución de la tarea 1 (en la carpeta
correspondiente).

Para que podamos ver sus tareas y corregir las mismas nos tienen que
hacer colaboradores de su repositorio de GitHub a Mauro (mauroloprete) y
a Natalia (natydasilva).

Utilicen el archivo .Rmd de esta tarea como base para la solución,
incorporando debajo de la pregunta su respuesta. Comienzá con los
ejercicios más sencillos y intentá ser ordenado/a, enumerá los
ejercicios y \textbf{utilizá un archivo .Rmd el cual debe compilar a
.pdf mostrando el código (chunks), o sea echo = TRUE}.

Verán que tal vez algunos ejercicios tienen alguna dificultad adicional
que las actividades, se espera que revisando el material sugerido en el
curso y leyendo la ayuda en R deberían ser capaces de resolver los
problemas. Si las preguntas no son suficientemente claras, pregunten en
el foro de EVA. Si las dudas no son de comprensión de la letra se
aconseja primero \textbf{buscar por su cuenta inicialmente} ya que es
parte del aprendizaje.

Si un ejercicio no lo pudiste realizar pero intentaste diferentes formas
\textbf{no} dejes en blanco el ejercicio, mantené el código junto al
razonamiento que utilizaste.

\newcommand{\bs}[1]{\boldsymbol{#1}}
\newcommand{\E}{\bs{\mathcal{E}}}
\newcommand{\F}{\bs{\mathcal{F}}}
\renewcommand{\v}{\bs{v}}
\renewcommand{\bfdefault}{m}

\newpage

\hypertarget{ejercicio-1-vectores}{%
\section{Ejercicio 1: Vectores}\label{ejercicio-1-vectores}}

\hypertarget{dado-los-siguientes-vectores-indicuxe1-a-quuxe9-tipo-de-vector-coercionan.}{%
\subsubsection{1. Dado los siguientes vectores, indicá a qué tipo de
vector
coercionan.}\label{dado-los-siguientes-vectores-indicuxe1-a-quuxe9-tipo-de-vector-coercionan.}}

\begin{Shaded}
\begin{Highlighting}[]
\NormalTok{w }\OtherTok{\textless{}{-}} \FunctionTok{c}\NormalTok{(}\DecValTok{29}\NormalTok{, 1L, }\StringTok{"22020202"}\NormalTok{, }\StringTok{"1"}\NormalTok{)}
\FunctionTok{cat}\NormalTok{(}\StringTok{"Cooerciona a"}\NormalTok{, }\FunctionTok{typeof}\NormalTok{(w))}
\end{Highlighting}
\end{Shaded}

\begin{verbatim}
## Cooerciona a character
\end{verbatim}

\begin{Shaded}
\begin{Highlighting}[]
\NormalTok{m }\OtherTok{\textless{}{-}} \FunctionTok{c}\NormalTok{(}\StringTok{"22+3"}\NormalTok{, }\DecValTok{33}\NormalTok{, }\ConstantTok{NA}\NormalTok{)}
\FunctionTok{cat}\NormalTok{(}\StringTok{"Cooerciona a"}\NormalTok{, }\FunctionTok{typeof}\NormalTok{(m))}
\end{Highlighting}
\end{Shaded}

\begin{verbatim}
## Cooerciona a character
\end{verbatim}

\begin{Shaded}
\begin{Highlighting}[]
\NormalTok{y }\OtherTok{\textless{}{-}} \FunctionTok{c}\NormalTok{(}\FunctionTok{seq}\NormalTok{(}\DecValTok{3}\SpecialCharTok{:}\DecValTok{25}\NormalTok{), 10L)}
\FunctionTok{cat}\NormalTok{(}\StringTok{"Cooerciona a"}\NormalTok{, }\FunctionTok{typeof}\NormalTok{(y))}
\end{Highlighting}
\end{Shaded}

\begin{verbatim}
## Cooerciona a integer
\end{verbatim}

\begin{Shaded}
\begin{Highlighting}[]
\NormalTok{z }\OtherTok{\textless{}{-}} \FunctionTok{paste}\NormalTok{(}\FunctionTok{seq}\NormalTok{(}\DecValTok{3}\SpecialCharTok{:}\DecValTok{25}\NormalTok{), 10L)}
\FunctionTok{cat}\NormalTok{(}\StringTok{"Cooerciona a"}\NormalTok{, }\FunctionTok{typeof}\NormalTok{(z))}
\end{Highlighting}
\end{Shaded}

\begin{verbatim}
## Cooerciona a character
\end{verbatim}

\hypertarget{almacenuxe1-los-vectores-de-la-parte-1-en-una-lista.}{%
\subsubsection{2. Almacená los vectores de la parte 1 en una
lista.}\label{almacenuxe1-los-vectores-de-la-parte-1-en-una-lista.}}

\begin{Shaded}
\begin{Highlighting}[]
\NormalTok{lista }\OtherTok{\textless{}{-}} \FunctionTok{list}\NormalTok{(w, m, y, z)}
\end{Highlighting}
\end{Shaded}

\hypertarget{creuxe1-una-funciuxf3n-que-intente-obtener-el-total-del-vector.-en-el-caso-que-no-pueda-cohercionar-a-vector-nuxfamerico-la-funciuxf3n-debe-de-contar-la-cantidad-de-elementos-diferentes-y-de-un-mensaje-de-advertencia.}{%
\subsubsection{3. Creá una función que intente obtener el total del
vector. En el caso que no pueda cohercionar a vector númerico, la
función debe de contar la cantidad de elementos diferentes y de un
mensaje de
advertencia.}\label{creuxe1-una-funciuxf3n-que-intente-obtener-el-total-del-vector.-en-el-caso-que-no-pueda-cohercionar-a-vector-nuxfamerico-la-funciuxf3n-debe-de-contar-la-cantidad-de-elementos-diferentes-y-de-un-mensaje-de-advertencia.}}

\begin{Shaded}
\begin{Highlighting}[]
\NormalTok{total }\OtherTok{\textless{}{-}} \ControlFlowTok{function}\NormalTok{(x)\{}
  \ControlFlowTok{if}\NormalTok{ (}\FunctionTok{is.numeric}\NormalTok{(x)) \{}
    \FunctionTok{return}\NormalTok{(}\FunctionTok{sum}\NormalTok{(x))}
\NormalTok{  \} }\ControlFlowTok{else}\NormalTok{ \{}
\NormalTok{    newX }\OtherTok{\textless{}{-}} \FunctionTok{as.numeric}\NormalTok{(x)}
    \ControlFlowTok{if}\NormalTok{ (}\SpecialCharTok{!}\FunctionTok{any}\NormalTok{(}\FunctionTok{is.na}\NormalTok{(newX))) \{}
      \FunctionTok{return}\NormalTok{(}\FunctionTok{sum}\NormalTok{(newX))}
\NormalTok{    \} }\ControlFlowTok{else}\NormalTok{ \{}
\NormalTok{      elemDif }\OtherTok{\textless{}{-}} \FunctionTok{length}\NormalTok{(}\FunctionTok{unique}\NormalTok{(x))}
      \FunctionTok{warning}\NormalTok{(}\FunctionTok{sprintf}\NormalTok{(}\StringTok{"No se pudo obtener el total del vector. Se encontraron \%d valores distintos."}\NormalTok{, elemDif))}
      \FunctionTok{return}\NormalTok{(}\ConstantTok{NA}\NormalTok{)}
\NormalTok{    \}}
\NormalTok{  \}}
\NormalTok{\}}
\end{Highlighting}
\end{Shaded}

\hypertarget{evalua-la-funciuxf3n-anterior-en-cada-elemento-de-la-lista-llamando-una-uxfanica-vez-la-funciuxf3n-creada-en-la-secciuxf3n-anterior.}{%
\subsubsection{4. Evalua la función anterior en cada elemento de la
lista llamando una única vez la función creada en la sección
anterior.}\label{evalua-la-funciuxf3n-anterior-en-cada-elemento-de-la-lista-llamando-una-uxfanica-vez-la-funciuxf3n-creada-en-la-secciuxf3n-anterior.}}

Pudes usar un \texttt{for\ loop} pero en este caso probá hacerlo de una
forma alternativa con la familia de funciones \texttt{apply}

\begin{Shaded}
\begin{Highlighting}[]
\FunctionTok{sapply}\NormalTok{(lista, total)}
\end{Highlighting}
\end{Shaded}

\begin{verbatim}
## [1] 22020233       NA      286       NA
\end{verbatim}

\hypertarget{cuuxe1l-es-la-diferencia-entre-c4-3-2-1-y-41}{%
\subsubsection{\texorpdfstring{5. ¿Cuál es la diferencia entre
\texttt{c(4,\ 3,\ 2,\ 1)} y
\texttt{4:1}?}{5. ¿Cuál es la diferencia entre c(4, 3, 2, 1) y 4:1?}}\label{cuuxe1l-es-la-diferencia-entre-c4-3-2-1-y-41}}

\begin{Shaded}
\begin{Highlighting}[]
\FunctionTok{c}\NormalTok{(}\DecValTok{4}\NormalTok{, }\DecValTok{3}\NormalTok{, }\DecValTok{2}\NormalTok{, }\DecValTok{1}\NormalTok{)}
\end{Highlighting}
\end{Shaded}

\begin{verbatim}
## [1] 4 3 2 1
\end{verbatim}

\begin{Shaded}
\begin{Highlighting}[]
\DecValTok{4}\SpecialCharTok{:}\DecValTok{1}
\end{Highlighting}
\end{Shaded}

\begin{verbatim}
## [1] 4 3 2 1
\end{verbatim}

Crean exactamente el mismo vector. Las formas de crearlo son distintas,
en el primero se explicito cada uno de los elementos, mientras que el
segundo se creo como una secuencia que va de 4 a 1 , con pasos de -1

\newpage

\hypertarget{ejercicio-2-factores}{%
\section{Ejercicio 2: Factores}\label{ejercicio-2-factores}}

\hypertarget{describa-brevemente-la-utilidad-de-trabajar-con-factores-y-los-posibles-inconvenientes-vistos-en-clase.}{%
\subsubsection{1. Describa brevemente la utilidad de trabajar con
factores y los posibles inconvenientes vistos en
clase.}\label{describa-brevemente-la-utilidad-de-trabajar-con-factores-y-los-posibles-inconvenientes-vistos-en-clase.}}

Dado el siguiente \texttt{factor} \texttt{x}:

\begin{Shaded}
\begin{Highlighting}[]
\NormalTok{x }\OtherTok{\textless{}{-}}
   \FunctionTok{factor}\NormalTok{(}
      \FunctionTok{c}\NormalTok{(}
         \StringTok{"a lto"}\NormalTok{,}
         \StringTok{" b a j o "}\NormalTok{,}
         \StringTok{"med i          o"}\NormalTok{,}
         \StringTok{"alt o"}\NormalTok{,}
         \StringTok{"muy alto"}\NormalTok{,}
         \StringTok{"bajo"}\NormalTok{,}
         \StringTok{"medio"}\NormalTok{,}
         \StringTok{"alto"}\NormalTok{,}
         \StringTok{"ALTO "}\NormalTok{,}
         \StringTok{"  MEDIO"}\NormalTok{,}
         \StringTok{"BAJO  "}\NormalTok{,}
         \StringTok{"MUY\_ ALTO "}\NormalTok{,}
         \StringTok{"muy muy muy alto"}\NormalTok{,}
         \StringTok{"no se que contestar"}\NormalTok{,}
         \StringTok{"Alto"}\NormalTok{,}
         \StringTok{"Ni idea"}\NormalTok{,}
         \StringTok{" muy alto "}\NormalTok{,}
         \StringTok{"alto, creo"}\NormalTok{,}
         \StringTok{"Bajo "}\NormalTok{,}
         \StringTok{"\_\_\_\_\_\_\_M \_\_\_\_\_\_"}\NormalTok{,}
         \StringTok{"GU    AU    "}\NormalTok{,}
         \StringTok{"GOL"}\NormalTok{,}
         \StringTok{"MUY BAJO!!!"}\NormalTok{,}
         \StringTok{"MUY BAJO                         "}\NormalTok{,}
         \StringTok{"MuY AlTo"}\NormalTok{,}
         \StringTok{"Me parece que alto"}\NormalTok{,}
         \StringTok{"demasiado bajo"}
\NormalTok{      )}
\NormalTok{   )}
\end{Highlighting}
\end{Shaded}

\hypertarget{generuxe1-un-nuevo-factor-llamalo-xx-transformando-el-objeto-x}{%
\subsubsection{\texorpdfstring{2. Generá un nuevo \texttt{factor}
(llamalo \texttt{xx}) transformando el objeto
\texttt{x}:}{2. Generá un nuevo factor (llamalo xx) transformando el objeto x:}}\label{generuxe1-un-nuevo-factor-llamalo-xx-transformando-el-objeto-x}}

Debes de seguir los siguientes pasos:

\begin{itemize}
\tightlist
\item
  Conviertan todos los valores a minúsculas mediante una función de R,
  NO lo hagan a mano.
\item
  Borren ``muy'' para que ``bajo'' y ``muy bajo'' sean iguales, pueden
  existir espacios, \textbf{hay que removerlos}.
\item
  Todo lo que no sea ``bajo'', ``medio'', ``alto'' debe ser excluido.
\item
  Las etiquetas deben ser : B bajo, M medio y A alto.
\item
  Recuerden (como vieron en las actividades) que pueden usar
  \emph{labels}.
\end{itemize}

El ejercicio tiene dos respuestas, una refleja un nivel básico-medio de
manipulación de cadenas de texto y el otro un nivel medio-avanzado. El
nivel básico-medio implica no poder extraer los niveles de
\texttt{\textquotesingle{}Me\ parece\ que\ alto\textquotesingle{}},
\texttt{\textquotesingle{}demasiado\ bajo\textquotesingle{}} y
\texttt{\textquotesingle{}alto,creo\textquotesingle{}}, el siguiente
nivel implica poder extraer los niveles \texttt{alto}, \texttt{bajo} y
\texttt{alto} respectivamente.

Si querés pasar de un nivel a otro podes buscar sobre regex \footnote{Visitá
  \url{https://regex101.com/}}, la función \texttt{base::regexpr} y
\texttt{base::regmatches}.

\textbf{Pistas}:

\begin{itemize}
\tightlist
\item
  Para el nivel básico-medio se obtiene la siguiente tabla de
  frecuencias:

  \begin{itemize}
  \tightlist
  \item
    alto: 10
  \item
    medio: 3
  \item
    bajo: 6
  \end{itemize}
\item
  Para el nivel avanzado, se le agregan al nivel alto dos cantidades y
  al bajo una observación.
\item
  Se deben corregir (y tomar en cuenta) todos los casos que contengan
  las palabras: bajo, medio, alto. Es decir, ``MUY ALTO'', ``ALTO''
  deben transformarse a ``alto'' y así sucesivamente.
\end{itemize}

\begin{Shaded}
\begin{Highlighting}[]
\NormalTok{x\_min }\OtherTok{\textless{}{-}} \FunctionTok{tolower}\NormalTok{(x)}
\NormalTok{x\_sin\_espacios }\OtherTok{\textless{}{-}} \FunctionTok{gsub}\NormalTok{(}\StringTok{\textquotesingle{} \textquotesingle{}}\NormalTok{, }\StringTok{\textquotesingle{}\textquotesingle{}}\NormalTok{, x\_min)}

\NormalTok{x\_bajo }\OtherTok{\textless{}{-}} \FunctionTok{grepl}\NormalTok{(}\StringTok{"bajo"}\NormalTok{, x\_sin\_espacios)}
\NormalTok{x\_medio }\OtherTok{\textless{}{-}} \FunctionTok{grepl}\NormalTok{(}\StringTok{"medio"}\NormalTok{, x\_sin\_espacios)}
\NormalTok{x\_alto }\OtherTok{\textless{}{-}} \FunctionTok{grepl}\NormalTok{(}\StringTok{"alto"}\NormalTok{, x\_sin\_espacios)}

\NormalTok{x[x\_bajo] }\OtherTok{\textless{}{-}} \FunctionTok{c}\NormalTok{(}\StringTok{\textquotesingle{}bajo\textquotesingle{}}\NormalTok{)}
\NormalTok{x[x\_medio] }\OtherTok{\textless{}{-}} \FunctionTok{c}\NormalTok{(}\StringTok{\textquotesingle{}medio\textquotesingle{}}\NormalTok{)}
\NormalTok{x[x\_alto] }\OtherTok{\textless{}{-}} \FunctionTok{c}\NormalTok{(}\StringTok{\textquotesingle{}alto\textquotesingle{}}\NormalTok{)}
\NormalTok{x[}\SpecialCharTok{!}\NormalTok{(x\_bajo }\SpecialCharTok{|}\NormalTok{ x\_medio }\SpecialCharTok{|}\NormalTok{ x\_alto)] }\OtherTok{\textless{}{-}} \FunctionTok{c}\NormalTok{(}\ConstantTok{NA}\NormalTok{)}


\NormalTok{x\_labels }\OtherTok{\textless{}{-}} \FunctionTok{c}\NormalTok{(}\StringTok{\textquotesingle{}A\textquotesingle{}}\NormalTok{, }\StringTok{\textquotesingle{}M\textquotesingle{}}\NormalTok{, }\StringTok{\textquotesingle{}B\textquotesingle{}}\NormalTok{)}
\NormalTok{x\_levels }\OtherTok{\textless{}{-}} \FunctionTok{c}\NormalTok{(}\StringTok{\textquotesingle{}alto\textquotesingle{}}\NormalTok{, }\StringTok{\textquotesingle{}medio\textquotesingle{}}\NormalTok{, }\StringTok{\textquotesingle{}bajo\textquotesingle{}}\NormalTok{)}

\NormalTok{xx }\OtherTok{\textless{}{-}} \FunctionTok{factor}\NormalTok{(x, }\AttributeTok{levels =}\NormalTok{ x\_levels , }\AttributeTok{labels =}\NormalTok{ x\_labels)}
\end{Highlighting}
\end{Shaded}

\hypertarget{generuxe1-el-siguiente-data.frame}{%
\subsubsection{\texorpdfstring{3. Generá el siguiente
\texttt{data.frame()}}{3. Generá el siguiente data.frame()}}\label{generuxe1-el-siguiente-data.frame}}

\begin{Shaded}
\begin{Highlighting}[]
\NormalTok{df }\OtherTok{\textless{}{-}} \FunctionTok{as.data.frame}\NormalTok{(}\FunctionTok{table}\NormalTok{(xx))}
\FunctionTok{names}\NormalTok{(df) }\OtherTok{\textless{}{-}} \FunctionTok{c}\NormalTok{(}\StringTok{"levels"}\NormalTok{, }\StringTok{"value"}\NormalTok{)}
\NormalTok{df[}\FunctionTok{c}\NormalTok{(}\DecValTok{2}\NormalTok{, }\DecValTok{3}\NormalTok{),] }\OtherTok{\textless{}{-}}\NormalTok{ df[}\FunctionTok{c}\NormalTok{(}\DecValTok{3}\NormalTok{, }\DecValTok{2}\NormalTok{),]}
\NormalTok{df}
\end{Highlighting}
\end{Shaded}

\begin{verbatim}
##   levels value
## 1      A    12
## 2      B     7
## 3      M     3
\end{verbatim}

\newpage

\hypertarget{ejercicio-3-listas}{%
\section{Ejercicio 3: Listas}\label{ejercicio-3-listas}}

\hypertarget{generuxe1-una-lista-que-se-llame-lista_t1-que-contenga}{%
\subsubsection{\texorpdfstring{1. Generá una lista que se llame
\texttt{lista\_t1} que
contenga:}{1. Generá una lista que se llame lista\_t1 que contenga:}}\label{generuxe1-una-lista-que-se-llame-lista_t1-que-contenga}}

\begin{itemize}
\tightlist
\item
  Un vector numérico de longitud 4 (\texttt{h}).
\item
  Una matriz de dimensión 4*3 (\texttt{u}).
\item
  La palabra ``chau'' (\texttt{palabra}).
\item
  Una secuencia diaria de fechas (clase Date) desde 2021/01/01 hasta
  2021/12/30 (\texttt{fecha}).
\end{itemize}

\begin{Shaded}
\begin{Highlighting}[]
\NormalTok{h }\OtherTok{\textless{}{-}} \FunctionTok{c}\NormalTok{(}\DecValTok{1}\NormalTok{, }\DecValTok{2}\NormalTok{, }\DecValTok{3}\NormalTok{, }\DecValTok{4}\NormalTok{)}
\NormalTok{u }\OtherTok{\textless{}{-}} \FunctionTok{matrix}\NormalTok{(}\DecValTok{1}\SpecialCharTok{:}\DecValTok{12}\NormalTok{, }\AttributeTok{nrow =} \DecValTok{4}\NormalTok{, }\AttributeTok{ncol =} \DecValTok{3}\NormalTok{)}
\NormalTok{palabra }\OtherTok{\textless{}{-}} \StringTok{"chau"}
\NormalTok{fecha }\OtherTok{\textless{}{-}} \FunctionTok{seq}\NormalTok{(}\FunctionTok{as.Date}\NormalTok{(}\StringTok{"2021{-}01{-}01"}\NormalTok{), }\FunctionTok{as.Date}\NormalTok{(}\StringTok{"2021{-}12{-}30"}\NormalTok{), }\AttributeTok{by =} \StringTok{"day"}\NormalTok{)}
\NormalTok{lista\_t1 }\OtherTok{\textless{}{-}} \FunctionTok{list}\NormalTok{(h, u, palabra, fecha)}
\end{Highlighting}
\end{Shaded}

\hypertarget{cuuxe1l-es-el-tercer-elemento-de-la-primera-fila-de-la-matriz-u-quuxe9-columna-lo-contiene}{%
\subsubsection{\texorpdfstring{2. ¿Cuál es el tercer elemento de la
primera fila de la matriz \texttt{u}? ¿Qué columna lo
contiene?}{2. ¿Cuál es el tercer elemento de la primera fila de la matriz u? ¿Qué columna lo contiene?}}\label{cuuxe1l-es-el-tercer-elemento-de-la-primera-fila-de-la-matriz-u-quuxe9-columna-lo-contiene}}

\begin{Shaded}
\begin{Highlighting}[]
\NormalTok{primera\_fila }\OtherTok{\textless{}{-}}\NormalTok{ u[}\DecValTok{1}\NormalTok{,]}
\NormalTok{primera\_fila[}\DecValTok{3}\NormalTok{] }\CommentTok{\# esta en la tercera columna}
\end{Highlighting}
\end{Shaded}

\begin{verbatim}
## [1] 9
\end{verbatim}

\hypertarget{cuuxe1l-es-la-diferencia-entre-hacer-lista_t12---0-y-lista_t12---0}{%
\subsubsection{\texorpdfstring{3. ¿Cuál es la diferencia entre hacer
\texttt{lista\_t1{[}{[}2{]}{]}{[}{]}\ \textless{}-\ 0} y
\texttt{lista\_t1{[}{[}2{]}{]}\ \textless{}-\ 0}?}{3. ¿Cuál es la diferencia entre hacer lista\_t1{[}{[}2{]}{]}{[}{]} \textless- 0 y lista\_t1{[}{[}2{]}{]} \textless- 0?}}\label{cuuxe1l-es-la-diferencia-entre-hacer-lista_t12---0-y-lista_t12---0}}

El primero reemplaza los valores de la matriz
\texttt{lista\_t1{[}{[}2{]}{]}}, el segundo reemplaza
\texttt{lista\_t1{[}{[}2{]}{]}} con 0

\hypertarget{ejercicio-4-matrices}{%
\section{Ejercicio 4: Matrices}\label{ejercicio-4-matrices}}

\hypertarget{generuxe1-una-matriz-a-de-dimensiuxf3n-43-y-una-matriz-b-de-dimensiuxf3n-42-con-nuxfameros-aleatorios-usando-alguna-funciuxf3n-predefinda-en-r.}{%
\subsubsection{\texorpdfstring{1. Generá una matriz \(A\) de dimensión
\(4*3\) y una matriz \(B\) de dimensión \(4*2\) con números aleatorios
usando alguna función predefinda en
R.}{1. Generá una matriz A de dimensión 4*3 y una matriz B de dimensión 4*2 con números aleatorios usando alguna función predefinda en R.}}\label{generuxe1-una-matriz-a-de-dimensiuxf3n-43-y-una-matriz-b-de-dimensiuxf3n-42-con-nuxfameros-aleatorios-usando-alguna-funciuxf3n-predefinda-en-r.}}

\begin{Shaded}
\begin{Highlighting}[]
\NormalTok{A }\OtherTok{\textless{}{-}} \FunctionTok{matrix}\NormalTok{(}\FunctionTok{sample}\NormalTok{(}\DecValTok{1}\SpecialCharTok{:}\DecValTok{100}\NormalTok{, }\DecValTok{12}\NormalTok{), }\AttributeTok{nrow =} \DecValTok{4}\NormalTok{, }\AttributeTok{ncol =} \DecValTok{3}\NormalTok{)}
\NormalTok{B }\OtherTok{\textless{}{-}} \FunctionTok{matrix}\NormalTok{(}\FunctionTok{sample}\NormalTok{(}\DecValTok{1}\SpecialCharTok{:}\DecValTok{100}\NormalTok{, }\DecValTok{8}\NormalTok{), }\AttributeTok{nrow =} \DecValTok{4}\NormalTok{, }\AttributeTok{ncol =} \DecValTok{2}\NormalTok{)}
\end{Highlighting}
\end{Shaded}

\hypertarget{calculuxe1-el-producto-elemento-a-elemento-de-la-primera-columna-de-la-matriz-a-por-la-uxfaltima-columna-de-la-matriz-b.}{%
\subsubsection{\texorpdfstring{2. Calculá el producto elemento a
elemento de la primera columna de la matriz \(A\) por la última columna
de la matriz
\(B\).}{2. Calculá el producto elemento a elemento de la primera columna de la matriz A por la última columna de la matriz B.}}\label{calculuxe1-el-producto-elemento-a-elemento-de-la-primera-columna-de-la-matriz-a-por-la-uxfaltima-columna-de-la-matriz-b.}}

\begin{Shaded}
\begin{Highlighting}[]
\NormalTok{A[,}\DecValTok{1}\NormalTok{] }\SpecialCharTok{*}\NormalTok{ B[,}\SpecialCharTok{{-}}\DecValTok{1}\NormalTok{]}
\end{Highlighting}
\end{Shaded}

\begin{verbatim}
## [1]  561 1462  456  504
\end{verbatim}

\hypertarget{calculuxe1-el-producto-matricial-entre-d-atb.-luego-seleccionuxe1-los-elementos-de-la-primer-y-tercera-fila-de-la-segunda-columna-en-un-paso.}{%
\subsubsection{\texorpdfstring{3. Calculá el producto matricial entre
\(D = A^TB\). Luego seleccioná los elementos de la primer y tercera fila
de la segunda columna (en un
paso).}{3. Calculá el producto matricial entre D = A\^{}TB. Luego seleccioná los elementos de la primer y tercera fila de la segunda columna (en un paso).}}\label{calculuxe1-el-producto-matricial-entre-d-atb.-luego-seleccionuxe1-los-elementos-de-la-primer-y-tercera-fila-de-la-segunda-columna-en-un-paso.}}

\begin{Shaded}
\begin{Highlighting}[]
\NormalTok{D }\OtherTok{\textless{}{-}} \FunctionTok{t}\NormalTok{(A) }\SpecialCharTok{\%*\%}\NormalTok{ B}
\NormalTok{D[}\FunctionTok{c}\NormalTok{(}\DecValTok{1}\NormalTok{, }\DecValTok{3}\NormalTok{), }\DecValTok{2}\NormalTok{]}
\end{Highlighting}
\end{Shaded}

\begin{verbatim}
## [1] 2983 9486
\end{verbatim}

\hypertarget{usuxe1-las-matrices-a-y-b-de-forma-tal-de-lograr-una-matriz-c-de-dimensiuxf3n-45.-con-la-funciuxf3n-attributes-inspeccionuxe1-los-atributos-de-c.-posteriormente-renombruxe1-filas-y-columnas-como-fila_1-fila_2columna_1-columna_2-vuelvuxe9-a-inspeccionar-los-atributos.-finalmente-generalizuxe1-y-escribuxed-una-funciuxf3n-que-reciba-como-argumento-una-matriz-y-devuelva-como-resultado-la-misma-matriz-con-columnas-y-filas-con-nombres.}{%
\subsubsection{\texorpdfstring{4. Usá las matrices \(A\) y \(B\) de
forma tal de lograr una matriz \(C\) de dimensión \(4*5\). Con la
función \texttt{attributes} inspeccioná los atributos de C.
Posteriormente renombrá filas y columnas como ``fila\_1'',
``fila\_2''\ldots{}``columna\_1'', ``columna\_2'', vuelvé a inspeccionar
los atributos. Finalmente, generalizá y escribí una función que reciba
como argumento una matriz y devuelva como resultado la misma matriz con
columnas y filas con
nombres.}{4. Usá las matrices A y B de forma tal de lograr una matriz C de dimensión 4*5. Con la función attributes inspeccioná los atributos de C. Posteriormente renombrá filas y columnas como ``fila\_1'', ``fila\_2''\ldots``columna\_1'', ``columna\_2'', vuelvé a inspeccionar los atributos. Finalmente, generalizá y escribí una función que reciba como argumento una matriz y devuelva como resultado la misma matriz con columnas y filas con nombres.}}\label{usuxe1-las-matrices-a-y-b-de-forma-tal-de-lograr-una-matriz-c-de-dimensiuxf3n-45.-con-la-funciuxf3n-attributes-inspeccionuxe1-los-atributos-de-c.-posteriormente-renombruxe1-filas-y-columnas-como-fila_1-fila_2columna_1-columna_2-vuelvuxe9-a-inspeccionar-los-atributos.-finalmente-generalizuxe1-y-escribuxed-una-funciuxf3n-que-reciba-como-argumento-una-matriz-y-devuelva-como-resultado-la-misma-matriz-con-columnas-y-filas-con-nombres.}}

\begin{Shaded}
\begin{Highlighting}[]
\NormalTok{  C }\OtherTok{\textless{}{-}} \FunctionTok{cbind}\NormalTok{(A, B)}
  \FunctionTok{attributes}\NormalTok{(C)}
\end{Highlighting}
\end{Shaded}

\begin{verbatim}
## $dim
## [1] 4 5
\end{verbatim}

\begin{Shaded}
\begin{Highlighting}[]
  \FunctionTok{rownames}\NormalTok{(C) }\OtherTok{\textless{}{-}} \FunctionTok{paste0}\NormalTok{(}\StringTok{"fila\_"}\NormalTok{, }\DecValTok{1}\SpecialCharTok{:}\DecValTok{4}\NormalTok{)}
  \FunctionTok{colnames}\NormalTok{(C) }\OtherTok{\textless{}{-}} \FunctionTok{paste0}\NormalTok{(}\StringTok{"columna\_"}\NormalTok{, }\DecValTok{1}\SpecialCharTok{:}\DecValTok{5}\NormalTok{)}
  \FunctionTok{attributes}\NormalTok{(C)}
\end{Highlighting}
\end{Shaded}

\begin{verbatim}
## $dim
## [1] 4 5
## 
## $dimnames
## $dimnames[[1]]
## [1] "fila_1" "fila_2" "fila_3" "fila_4"
## 
## $dimnames[[2]]
## [1] "columna_1" "columna_2" "columna_3" "columna_4" "columna_5"
\end{verbatim}

\begin{Shaded}
\begin{Highlighting}[]
\NormalTok{    renombrar\_matriz }\OtherTok{\textless{}{-}} \ControlFlowTok{function}\NormalTok{(matriz) \{}
      \FunctionTok{rownames}\NormalTok{(matriz) }\OtherTok{\textless{}{-}} \FunctionTok{paste0}\NormalTok{(}\StringTok{"fila\_"}\NormalTok{, }\DecValTok{1}\SpecialCharTok{:}\FunctionTok{nrow}\NormalTok{(matriz))}
      \FunctionTok{colnames}\NormalTok{(matriz) }\OtherTok{\textless{}{-}} \FunctionTok{paste0}\NormalTok{(}\StringTok{"columna\_"}\NormalTok{, }\DecValTok{1}\SpecialCharTok{:}\FunctionTok{ncol}\NormalTok{(matriz))}
      \FunctionTok{return}\NormalTok{(matriz)}
\NormalTok{    \}}
    \FunctionTok{renombrar\_matriz}\NormalTok{(C)}
\end{Highlighting}
\end{Shaded}

\begin{verbatim}
##        columna_1 columna_2 columna_3 columna_4 columna_5
## fila_1        11        20        56        81        51
## fila_2        43        93        61        65        34
## fila_3        19        54        79        27        24
## fila_4        18        72        95        13        28
\end{verbatim}

\hypertarget{puntos-extra-genelarizuxe1-la-funciuxf3n-para-que-funcione-con-arrays-de-forma-que-renombre-filas-columnas-y-matrices.}{%
\subsubsection{\texorpdfstring{5. \textbf{Puntos Extra}: genelarizá la
función para que funcione con arrays de forma que renombre filas,
columnas y
matrices.}{5. Puntos Extra: genelarizá la función para que funcione con arrays de forma que renombre filas, columnas y matrices.}}\label{puntos-extra-genelarizuxe1-la-funciuxf3n-para-que-funcione-con-arrays-de-forma-que-renombre-filas-columnas-y-matrices.}}

\newpage

\hypertarget{ejercicio-5-estructuras-de-control-y-funciones-vectrorizadas}{%
\section{Ejercicio 5: Estructuras de control y funciones
vectrorizadas}\label{ejercicio-5-estructuras-de-control-y-funciones-vectrorizadas}}

\hypertarget{quuxe9-hace-la-funciuxf3n-ifelse-del-paquete-base-de-r}{%
\subsubsection{\texorpdfstring{1. ¿Qué hace la función \texttt{ifelse()}
del paquete \texttt{base} de
R?}{1. ¿Qué hace la función ifelse() del paquete base de R?}}\label{quuxe9-hace-la-funciuxf3n-ifelse-del-paquete-base-de-r}}

Dada una condicion, cuando es TRUE devuelve un valor, y cuando es FALSE
devuelve otro. Es vectorizada, por lo que si la condicion es
vectorizada, la respuesta va a ser un vector con la respuesta de cada
elemento del vector

\hypertarget{dado-el-vector-x-tal-que-x---c8-6-22-1-0--2--45-utilizando-la-funciuxf3n-ifelse-del-paquete-base-reemplazuxe1-todos-los-elementos-mayores-estrictos-a-0-por-1-y-todos-los-elementos-menores-o-iguales-a-0-por-0.}{%
\subsubsection{\texorpdfstring{2. Dado el vector \(x\) tal que:
\texttt{x\ \textless{}-\ c(8,\ 6,\ 22,\ 1,\ 0,\ -2,\ -45)}, utilizando
la función \texttt{ifelse()} del paquete \texttt{base}, reemplazá todos
los elementos mayores estrictos a \texttt{0} por \texttt{1}, y todos los
elementos menores o iguales a \texttt{0} por
\texttt{0}.}{2. Dado el vector x tal que: x \textless- c(8, 6, 22, 1, 0, -2, -45), utilizando la función ifelse() del paquete base, reemplazá todos los elementos mayores estrictos a 0 por 1, y todos los elementos menores o iguales a 0 por 0.}}\label{dado-el-vector-x-tal-que-x---c8-6-22-1-0--2--45-utilizando-la-funciuxf3n-ifelse-del-paquete-base-reemplazuxe1-todos-los-elementos-mayores-estrictos-a-0-por-1-y-todos-los-elementos-menores-o-iguales-a-0-por-0.}}

\begin{Shaded}
\begin{Highlighting}[]
\NormalTok{x }\OtherTok{\textless{}{-}} \FunctionTok{c}\NormalTok{(}\DecValTok{8}\NormalTok{, }\DecValTok{6}\NormalTok{, }\DecValTok{22}\NormalTok{, }\DecValTok{1}\NormalTok{, }\DecValTok{0}\NormalTok{, }\SpecialCharTok{{-}}\DecValTok{2}\NormalTok{, }\SpecialCharTok{{-}}\DecValTok{45}\NormalTok{)}
\FunctionTok{ifelse}\NormalTok{(x }\SpecialCharTok{\textgreater{}} \DecValTok{0}\NormalTok{, }\DecValTok{1}\NormalTok{, }\DecValTok{0}\NormalTok{)}
\end{Highlighting}
\end{Shaded}

\begin{verbatim}
## [1] 1 1 1 1 0 0 0
\end{verbatim}

\hypertarget{por-quuxe9-no-fuuxe9-necesario-usar-un-loop}{%
\subsubsection{3. ¿Por qué no fué necesario usar un loop
?}\label{por-quuxe9-no-fuuxe9-necesario-usar-un-loop}}

Porque la función \texttt{ifelse()} es vectorizada

\hypertarget{quuxe9-es-un-while-loop-y-cuxf3mo-es-la-estructura-para-generar-uno-en-r-en-quuxe9-se-diferencia-de-un-for-loop}{%
\subsubsection{4. ¿Qué es un while loop y cómo es la estructura para
generar uno en R? ¿En qué se diferencia de un for
loop?}\label{quuxe9-es-un-while-loop-y-cuxf3mo-es-la-estructura-para-generar-uno-en-r-en-quuxe9-se-diferencia-de-un-for-loop}}

Un while loop es una estructura de control que sirve para iterar
mientras una condición sea verdadera. La estructura es la siguiente:

\begin{Shaded}
\begin{Highlighting}[]
\ControlFlowTok{while}\NormalTok{(condicion)\{}
  \CommentTok{\# codigo a ejecutar}
\NormalTok{\}}
\end{Highlighting}
\end{Shaded}

\hypertarget{dada-la-estructura-siguiente}{%
\subsubsection{5. Dada la estructura
siguiente:}\label{dada-la-estructura-siguiente}}

\begin{Shaded}
\begin{Highlighting}[]
\NormalTok{x }\OtherTok{\textless{}{-}} \FunctionTok{c}\NormalTok{(}\DecValTok{1}\NormalTok{,}\DecValTok{2}\NormalTok{,}\DecValTok{3}\NormalTok{)}
\NormalTok{suma }\OtherTok{\textless{}{-}} \DecValTok{0}
\NormalTok{i }\OtherTok{\textless{}{-}} \DecValTok{1}
\ControlFlowTok{while}\NormalTok{(i }\SpecialCharTok{\textless{}} \DecValTok{6}\NormalTok{)\{}
\NormalTok{ suma }\OtherTok{=}\NormalTok{ suma }\SpecialCharTok{+}\NormalTok{ x[i]      }
\NormalTok{ i }\OtherTok{\textless{}{-}}\NormalTok{ i }\SpecialCharTok{+} \DecValTok{1}     
\NormalTok{\}}
\end{Highlighting}
\end{Shaded}

\hypertarget{modificuxe1-la-estructura-anterior-para-que-suma-valga-0-si-el-vector-tiene-largo-menor-a-5-o-que-sume-los-primeros-5-elementos-si-el-vector-tiene-largo-mayor-a-5.-a-partir-de-ella-generuxe1-una-funciuxf3n-que-se-llame-sumar_si-y-verificuxe1-que-funcione-utilizando-los-vectores-y---c13-z---c115.}{%
\subsubsection{\texorpdfstring{6. Modificá la estructura anterior para
que \texttt{suma} valga 0 si el vector tiene largo menor a 5, o que sume
los primeros 5 elementos si el vector tiene largo mayor a 5. A partir de
ella generá una función que se llame \texttt{sumar\_si} y verificá que
funcione utilizando los vectores \texttt{y\ \textless{}-\ c(1:3)},
\texttt{z\ \textless{}-\ c(1:15)}.}{6. Modificá la estructura anterior para que suma valga 0 si el vector tiene largo menor a 5, o que sume los primeros 5 elementos si el vector tiene largo mayor a 5. A partir de ella generá una función que se llame sumar\_si y verificá que funcione utilizando los vectores y \textless- c(1:3), z \textless- c(1:15).}}\label{modificuxe1-la-estructura-anterior-para-que-suma-valga-0-si-el-vector-tiene-largo-menor-a-5-o-que-sume-los-primeros-5-elementos-si-el-vector-tiene-largo-mayor-a-5.-a-partir-de-ella-generuxe1-una-funciuxf3n-que-se-llame-sumar_si-y-verificuxe1-que-funcione-utilizando-los-vectores-y---c13-z---c115.}}

\begin{Shaded}
\begin{Highlighting}[]
\NormalTok{sumar\_si }\OtherTok{\textless{}{-}} \ControlFlowTok{function}\NormalTok{(x) \{}
  
  \ControlFlowTok{if}\NormalTok{(}\FunctionTok{length}\NormalTok{(x) }\SpecialCharTok{\textless{}} \DecValTok{5}\NormalTok{) }\FunctionTok{return}\NormalTok{(}\DecValTok{0}\NormalTok{)}
  
\NormalTok{  suma }\OtherTok{\textless{}{-}} \DecValTok{0}
\NormalTok{  i }\OtherTok{\textless{}{-}} \DecValTok{1}
  \ControlFlowTok{while}\NormalTok{(i }\SpecialCharTok{\textless{}} \DecValTok{6}\NormalTok{ )\{}
\NormalTok{    suma }\OtherTok{=}\NormalTok{ suma }\SpecialCharTok{+}\NormalTok{ x[i]}
\NormalTok{    i }\OtherTok{\textless{}{-}}\NormalTok{ i }\SpecialCharTok{+} \DecValTok{1}
\NormalTok{  \}}
  \FunctionTok{return}\NormalTok{(suma)}
\NormalTok{\}}

\NormalTok{y }\OtherTok{\textless{}{-}} \FunctionTok{c}\NormalTok{(}\DecValTok{1}\SpecialCharTok{:}\DecValTok{3}\NormalTok{)}
\NormalTok{z }\OtherTok{\textless{}{-}} \FunctionTok{c}\NormalTok{(}\DecValTok{1}\SpecialCharTok{:}\DecValTok{15}\NormalTok{)}

\FunctionTok{sumar\_si}\NormalTok{(y)}
\end{Highlighting}
\end{Shaded}

\begin{verbatim}
## [1] 0
\end{verbatim}

\begin{Shaded}
\begin{Highlighting}[]
\FunctionTok{sumar\_si}\NormalTok{(z)}
\end{Highlighting}
\end{Shaded}

\begin{verbatim}
## [1] 15
\end{verbatim}

\hypertarget{intentuxe1-escribir-en-funciuxf3n-de-forma-vectorizada.}{%
\subsubsection{7. Intentá escribir en función de forma
vectorizada.}\label{intentuxe1-escribir-en-funciuxf3n-de-forma-vectorizada.}}

\begin{Shaded}
\begin{Highlighting}[]
\NormalTok{sumar\_si\_vectorizado }\OtherTok{\textless{}{-}} \ControlFlowTok{function}\NormalTok{(x) \{}
  \ControlFlowTok{if}\NormalTok{(}\FunctionTok{length}\NormalTok{(x) }\SpecialCharTok{\textless{}} \DecValTok{5}\NormalTok{) }\FunctionTok{return}\NormalTok{(}\DecValTok{0}\NormalTok{)}
  \FunctionTok{return}\NormalTok{(}\FunctionTok{sum}\NormalTok{(x[}\DecValTok{1}\SpecialCharTok{:}\DecValTok{5}\NormalTok{]))}
\NormalTok{\}}

\FunctionTok{sumar\_si\_vectorizado}\NormalTok{(y)}
\end{Highlighting}
\end{Shaded}

\begin{verbatim}
## [1] 0
\end{verbatim}

\begin{Shaded}
\begin{Highlighting}[]
\FunctionTok{sumar\_si\_vectorizado}\NormalTok{(z)}
\end{Highlighting}
\end{Shaded}

\begin{verbatim}
## [1] 15
\end{verbatim}

Probá que la función devuelva el mismo resultado usando
\texttt{all.equal}.

\begin{Shaded}
\begin{Highlighting}[]
\FunctionTok{all.equal}\NormalTok{(}\FunctionTok{sumar\_si}\NormalTok{(y), }\FunctionTok{sumar\_si\_vectorizado}\NormalTok{(y))}
\end{Highlighting}
\end{Shaded}

\begin{verbatim}
## [1] TRUE
\end{verbatim}

\begin{Shaded}
\begin{Highlighting}[]
\FunctionTok{all.equal}\NormalTok{(}\FunctionTok{sumar\_si}\NormalTok{(z), }\FunctionTok{sumar\_si\_vectorizado}\NormalTok{(z))}
\end{Highlighting}
\end{Shaded}

\begin{verbatim}
## [1] TRUE
\end{verbatim}

\hypertarget{ejercicio-6-ordenar}{%
\section{Ejercicio 6: Ordenar}\label{ejercicio-6-ordenar}}

\hypertarget{generuxe1-una-funciuxf3n-ordenar_xque-para-cualquier-vector-numuxe9rico-ordene-sus-elementos-de-menor-a-mayor.-por-ejemplo}{%
\subsubsection{\texorpdfstring{1. Generá una función
\texttt{ordenar\_x()}que para cualquier vector numérico, ordene sus
elementos de menor a mayor. Por
ejemplo:}{1. Generá una función ordenar\_x()que para cualquier vector numérico, ordene sus elementos de menor a mayor. Por ejemplo:}}\label{generuxe1-una-funciuxf3n-ordenar_xque-para-cualquier-vector-numuxe9rico-ordene-sus-elementos-de-menor-a-mayor.-por-ejemplo}}

Sea \texttt{x\ \textless{}-\ c(3,4,5,-2,1)}, \texttt{ordenar\_x(x)}
devuelve \texttt{c(-2,1,3,4,5)}.

Para controlar, generá dos vectores numéricos cualquiera y pasalos como
argumentos en \texttt{ordenar\_x()}.

Observación: Si usa la función \texttt{base::order()} entonces debe
escribir 2 funciones. Una usando \texttt{base::order()} y otra sin
usarla.

\begin{Shaded}
\begin{Highlighting}[]
\NormalTok{ordenar\_x }\OtherTok{\textless{}{-}} \ControlFlowTok{function}\NormalTok{(x) \{}
  \ControlFlowTok{if}\NormalTok{ (}\FunctionTok{length}\NormalTok{(x) }\SpecialCharTok{\textless{}=} \DecValTok{1}\NormalTok{) }\FunctionTok{return}\NormalTok{(x)}
  \ControlFlowTok{else}\NormalTok{ \{}
\NormalTok{    pivot }\OtherTok{\textless{}{-}}\NormalTok{ x[}\DecValTok{1}\NormalTok{]}
\NormalTok{    left }\OtherTok{\textless{}{-}}\NormalTok{ x[}\SpecialCharTok{{-}}\DecValTok{1}\NormalTok{][x[}\SpecialCharTok{{-}}\DecValTok{1}\NormalTok{] }\SpecialCharTok{\textless{}=}\NormalTok{ pivot]}
\NormalTok{    right }\OtherTok{\textless{}{-}}\NormalTok{ x[}\SpecialCharTok{{-}}\DecValTok{1}\NormalTok{][x[}\SpecialCharTok{{-}}\DecValTok{1}\NormalTok{] }\SpecialCharTok{\textgreater{}}\NormalTok{ pivot]}
    \FunctionTok{return}\NormalTok{(}\FunctionTok{c}\NormalTok{(}\FunctionTok{ordenar\_x}\NormalTok{(left), pivot, }\FunctionTok{ordenar\_x}\NormalTok{(right)))}
\NormalTok{  \}}
\NormalTok{\}}


\NormalTok{ordenar\_x2 }\OtherTok{\textless{}{-}} \ControlFlowTok{function}\NormalTok{(x) \{}
  \FunctionTok{return}\NormalTok{(x[}\FunctionTok{order}\NormalTok{(x)])}
\NormalTok{\}}

\NormalTok{x }\OtherTok{\textless{}{-}} \FunctionTok{c}\NormalTok{(}\DecValTok{3}\NormalTok{,}\DecValTok{4}\NormalTok{,}\DecValTok{5}\NormalTok{,}\SpecialCharTok{{-}}\DecValTok{2}\NormalTok{,}\DecValTok{1}\NormalTok{)}
\NormalTok{y }\OtherTok{\textless{}{-}} \FunctionTok{c}\NormalTok{(}\DecValTok{1}\NormalTok{, }\DecValTok{2}\NormalTok{, }\DecValTok{3}\NormalTok{, }\DecValTok{4}\NormalTok{, }\DecValTok{5}\NormalTok{, }\DecValTok{6}\NormalTok{, }\DecValTok{7}\NormalTok{, }\DecValTok{8}\NormalTok{, }\DecValTok{9}\NormalTok{, }\DecValTok{10}\NormalTok{)}
\FunctionTok{ordenar\_x}\NormalTok{(x)}
\end{Highlighting}
\end{Shaded}

\begin{verbatim}
## [1] -2  1  3  4  5
\end{verbatim}

\begin{Shaded}
\begin{Highlighting}[]
\FunctionTok{ordenar\_x2}\NormalTok{(x)}
\end{Highlighting}
\end{Shaded}

\begin{verbatim}
## [1] -2  1  3  4  5
\end{verbatim}

\begin{Shaded}
\begin{Highlighting}[]
\FunctionTok{ordenar\_x}\NormalTok{(y)}
\end{Highlighting}
\end{Shaded}

\begin{verbatim}
##  [1]  1  2  3  4  5  6  7  8  9 10
\end{verbatim}

\begin{Shaded}
\begin{Highlighting}[]
\FunctionTok{ordenar\_x2}\NormalTok{(y)}
\end{Highlighting}
\end{Shaded}

\begin{verbatim}
##  [1]  1  2  3  4  5  6  7  8  9 10
\end{verbatim}

\hypertarget{quuxe9-devuelve-orderorderx}{%
\subsubsection{\texorpdfstring{2. ¿Qué devuelve
\texttt{order(order(x))}?}{2. ¿Qué devuelve order(order(x))?}}\label{quuxe9-devuelve-orderorderx}}

\texttt{order(order(x))} devuelve el lugar que ocuparia cada elemento de
x si estuviera ordenado de menor a mayor

\newpage

\hypertarget{ejercicio-7-argumentos-de-funciones}{%
\section{Ejercicio 7: Argumentos de
funciones}\label{ejercicio-7-argumentos-de-funciones}}

\hypertarget{quuxe9-funciuxf3n-del-paquete-base-es-la-que-tiene-mayor-cantidad-de-argumentos}{%
\subsubsection{- ¿Qué función del paquete base es la que tiene mayor
cantidad de
argumentos?}\label{quuxe9-funciuxf3n-del-paquete-base-es-la-que-tiene-mayor-cantidad-de-argumentos}}

\textbf{Pistas}: Posible solución:

\begin{enumerate}
\def\labelenumi{\arabic{enumi}.}
\setcounter{enumi}{-1}
\tightlist
\item
  Argumentos = \texttt{formals()}
\item
  Para comenzar use \texttt{ls("package:base")} y luego revise la
  función \texttt{get()} y \texttt{mget()} (use esta última, necesita
  modificar un parámetro ó formals).
\item
  Revise la funcion Filter
\item
  Itere
\item
  Obtenga el índice del valor máximo
\end{enumerate}

\begin{Shaded}
\begin{Highlighting}[]
\NormalTok{  lista\_base }\OtherTok{\textless{}{-}} \FunctionTok{ls}\NormalTok{(}\StringTok{"package:base"}\NormalTok{)}
\NormalTok{  cant\_arg }\OtherTok{\textless{}{-}} \FunctionTok{sapply}\NormalTok{(lista\_base, }\ControlFlowTok{function}\NormalTok{(name\_f)\{}
\NormalTok{    fun }\OtherTok{\textless{}{-}} \FunctionTok{get}\NormalTok{(name\_f)}
    \ControlFlowTok{if}\NormalTok{(}\FunctionTok{is.function}\NormalTok{(fun)) \{}
      \FunctionTok{return}\NormalTok{(}\FunctionTok{length}\NormalTok{(}\FunctionTok{formals}\NormalTok{(fun)))}
\NormalTok{    \} }\ControlFlowTok{else}\NormalTok{ \{}
      \FunctionTok{print}\NormalTok{(name\_f)}
      \FunctionTok{return}\NormalTok{(}\ConstantTok{NA}\NormalTok{)}
\NormalTok{    \}}
\NormalTok{  \})}
\end{Highlighting}
\end{Shaded}

\begin{verbatim}
## [1] "F"
## [1] "letters"
## [1] "LETTERS"
## [1] "month.abb"
## [1] "month.name"
## [1] "pi"
## [1] "R.version"
## [1] "R.version.string"
## [1] "T"
## [1] "version"
\end{verbatim}

\begin{Shaded}
\begin{Highlighting}[]
\NormalTok{  indice\_max }\OtherTok{\textless{}{-}} \FunctionTok{which.max}\NormalTok{(cant\_arg)}
\NormalTok{  lista\_base[indice\_max]}
\end{Highlighting}
\end{Shaded}

\begin{verbatim}
## [1] "scan"
\end{verbatim}

\hypertarget{ejercicio-8-factores}{%
\section{Ejercicio 8: Factores}\label{ejercicio-8-factores}}

Dado el siguiente \texttt{factor}

\begin{Shaded}
\begin{Highlighting}[]
\NormalTok{f1 }\OtherTok{\textless{}{-}} \FunctionTok{factor}\NormalTok{(letters)}
\end{Highlighting}
\end{Shaded}

\hypertarget{quuxe9-hace-el-siguiente-cuxf3digo-explicuxe1-las-diferencias-o-semejanzas.}{%
\subsubsection{- ¿Qué hace el siguiente código? Explicá las diferencias
o
semejanzas.}\label{quuxe9-hace-el-siguiente-cuxf3digo-explicuxe1-las-diferencias-o-semejanzas.}}

\begin{Shaded}
\begin{Highlighting}[]
\FunctionTok{levels}\NormalTok{(f1) }\OtherTok{\textless{}{-}} \FunctionTok{rev}\NormalTok{(}\FunctionTok{levels}\NormalTok{(f1))}
\NormalTok{f2 }\OtherTok{\textless{}{-}} \FunctionTok{rev}\NormalTok{(}\FunctionTok{factor}\NormalTok{(letters))}
\NormalTok{f3 }\OtherTok{\textless{}{-}} \FunctionTok{factor}\NormalTok{(letters, }\AttributeTok{levels =} \FunctionTok{rev}\NormalTok{(letters)) }
\end{Highlighting}
\end{Shaded}

El primero invierte el orden de los niveles del factor, no modifica el
orden de los elementos del vector. El segundo invierte el orden de los
elementos del vector, no modifica el orden de los niveles del factor. El
tercero invierte el orden de los niveles del factor y el orden de los
elementos del vector.

\end{document}
